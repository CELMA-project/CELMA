% !TEX TS-program = pdflatex
% !TEX encoding = UTF-8 Unicode

% This is a simple template for a LaTeX document using the "article" class.
% See "book", "report", "letter" for other types of document.

\documentclass[11pt]{article} % use larger type; default would be 10pt

\usepackage[utf8]{inputenc} % set input encoding (not needed with XeLaTeX)

%%% Examples of Article customizations
% These packages are optional, depending whether you want the features they provide.
% See the LaTeX Companion or other references for full information.

%%% PAGE DIMENSIONS
\usepackage{geometry} % to change the page dimensions
\geometry{a4paper} % or letterpaper (US) or a5paper or....
% \geometry{margin=2in} % for example, change the margins to 2 inches all round
% \geometry{landscape} % set up the page for landscape
%   read geometry.pdf for detailed page layout information

\usepackage{graphicx} % support the \includegraphics command and options

% \usepackage[parfill]{parskip} % Activate to begin paragraphs with an empty line rather than an indent

%%% PACKAGES
\usepackage{booktabs} % for much better looking tables
\usepackage{array} % for better arrays (eg matrices) in maths
\usepackage{paralist} % very flexible & customisable lists (eg. enumerate/itemize, etc.)
\usepackage{verbatim} % adds environment for commenting out blocks of text & for better verbatim
\usepackage{subfig} % make it possible to include more than one captioned figure/table in a single float
% These packages are all incorporated in the memoir class to one degree or another...

%%% HEADERS & FOOTERS
\usepackage{fancyhdr} % This should be set AFTER setting up the page geometry
\pagestyle{fancy} % options: empty , plain , fancy
\renewcommand{\headrulewidth}{0pt} % customise the layout...
\lhead{}\chead{}\rhead{}
\lfoot{}\cfoot{\thepage}\rfoot{}

%%% SECTION TITLE APPEARANCE
\usepackage{sectsty}
\allsectionsfont{\sffamily\mdseries\upshape} % (See the fntguide.pdf for font help)
% (This matches ConTeXt defaults)

%%% ToC (table of contents) APPEARANCE
\usepackage[nottoc,notlof,notlot]{tocbibind} % Put the bibliography in the ToC
\usepackage[titles,subfigure]{tocloft} % Alter the style of the Table of Contents
\renewcommand{\cftsecfont}{\rmfamily\mdseries\upshape}
\renewcommand{\cftsecpagefont}{\rmfamily\mdseries\upshape} % No bold!

%%% END Article customizations

%%% The "real" document content comes below...

\title{Cyto equations}
\author{Volker Naulin}
%\date{} % Activate to display a given date or no date (if empty),
         % otherwise the current date is printed 

\begin{document}
\maketitle

\section{Braginskii equations}
Continuity equation for electrons, assuming $n_e = n_i$:
\begin{equation}
\frac{\partial n}{\partial t} + \nabla \cdot (n \vec V_{e}) = 0
\end{equation}
Ion momentum equation
\begin{equation}
M n \frac{d \vec V_{i}}{d t} + \nabla\cdot {\bf P_{i}} 
- Z e n \left( \vec E + \frac{1}{c} \vec V_{i} \times \vec B\right) = F_{i}
\end{equation}
Electron momentum equation
\begin{equation}
m n \frac{d \vec V_{e}}{d t} + \nabla\cdot {\bf P_{e}} 
+ e n \left( \vec E + \frac{1}{c} \vec V_{e} \times \vec B\right) = F_{e}
\end{equation}
Electron temperature equation:
\begin{equation}
\frac{3}{2} n  \frac{d  T_{e}}{d t} + p_e \nabla\cdot \vec V_{e} =  
-\nabla \cdot \vec q_e + Q_e
\end{equation}

Here ion and electron temperature are assumed equal, current only important in the parallel direction
$$
Q_e = +R \frac{j}{ne} = n e j \nu \frac{j}{ne} -0.71 n \nabla_\|(kT_e)  \frac{j}{ne}
$$
for which we arrive at 
$$
Q_e  = \nu j^2 -0.71  \nabla_\|(kT_e)  \frac{j}{e}
$$
So in the temperature equation we get after dividing by $\frac{3}{2} nT$ and introducing $LT = \log T_e$

\begin{equation}
\frac{d  LT}{d t} =-\frac{2}{3} \nabla\cdot \vec V_{\|,e}   
-\frac{2}{3} \frac{\nabla \cdot \vec q_e}{nT} +\frac{2}{3}  \nu \frac{j}{T}\frac{j}{n} -0.71 \frac{2}{3}  \nabla_\| LT  \frac{j}{n}
\end{equation}

\section{Parameters}

\section{Numerics}

\section{Results}


\section{Neutrals}

We use the neutrals fluxes as given by 
$$
\Gamma = \left< \sigma \nu \right> n_1 n_2 
$$
and more specific use CF. Barnett, Atomic Data for Fusion Vol 1, ,
ORNL-6086/V1 for cahrge exchange and Goldston Rutherford for
Ionisation and recombination.


Ionisation:
\begin{equation}
\Gamma_{ion} = \frac{2 \star 10^{-13}}{6 + T_e/\Phi_{ion}}
\left(\frac{\Phi_{ion}}{T_e}\right)^{1/2} e^{-\Phi_{ion}/T_e} n_i n_e \mbox{m}^{-3} \mbox{s}^{-1} 
\end{equation}

Recombination:
\begin{equation}
\Gamma_{rec} = 7 \star 10^{-20}
\left(\frac{\Phi_{ion}}{T_e}\right)^{1/2} n_i n_i\mbox{m}^{-3} \mbox{s}^{-1} 
\end{equation}

Charge exchange:
\begin{equation}
\Gamma_{cx} = 3 \star 10^{-19}
T_i n_i n_n\mbox{m}^{-3} \mbox{s}^{-1} 
\end{equation}

with the latter expression valid in the absense of a relative drift
between neutrals and ions (or: Thermal energy larger than energy in
relative motion).


In the code we calculate these rates usen $n_i = n_e$ and normalise
the rates to $n_e$

At present charge exchange is not used. 


\subsection{Basic test}

A first test is just to run the code with the background pressure
neutral gas and its measured temperature. Ionisation is not sufficient
to keep the density, but we also know that the helicon waves are the
drivers of density. This should be different in a 2D situation? 


A second test checking charge excahnge as a passive diagnostic.


\subsection{Strategy}

Insert a trivial feedback on the neutrals, wiithout neutral dynamics.

Introduce a hot neutral species, with charge exchange taking neutrals
from cold to hot.

Introduce neutral interaction in higher fluid moments, for now we have
the particle source sink and the ion current by cahrge exchange
(Pedersen current) in the vorticity equation. But there are additional
terms in Ohms law and temperature equation.

Introduce a neutral dynamical model


\end{document}
